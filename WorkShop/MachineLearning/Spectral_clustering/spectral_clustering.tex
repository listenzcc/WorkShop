\documentclass[a4paper]{article}
\usepackage{amsmath}

\begin{document}

\abstract
In a graph, known as $G(V, E)$, an important problem is how to clustering the vertex \emph{automatically}.
That means that the clustering method should be \emph{unsupervised}.
The spectral clustering method is a powerful solution.

\section {Laplacian matrix and graph}
The weight matrix is defined as
\begin{equation*}
    W = [w_{ij}]
\end{equation*}
in which, $w_{ij}=w_{ji}$ is the measurement weight of the edge, refers the distance between $V_i$ and $V_j$.

The degree matrix is diagonal matrix
\begin{equation*}
    d_{ii} = \sum_{j=1}^{N}w_{ij}
\end{equation*}
where $N$ is the number of vertex.

The subtraction is \emph{Laplacian} matrix $L$
\begin{equation}
    \label{equation: Laplacian matrix}
    L = D - W
\end{equation}

For any $N$ dimensional vector $f \in R^N$
\begin{equation}
    \label{equation: Laplacian matrix times vector}
    f^T L f = \frac{1}{2} \sum_{i, j=1}^{N} w_{ij}(f_i - f_j)^2
\end{equation}

\section{Ratio Cut}
For graph $G(V, E)$, separate vertex into $k$ sets $A_1, A_2, \cdots , A_k$, the union is $V$.
For sets of $A$ and $B$, the weight between them is
\begin{equation*}
    W(A, B) = \sum_{i \in A, j \in B} w_{ij}
\end{equation*}

For a cut protocol $cut$, the weight is defined as
\begin{equation}
    \label{equation:Loss of cut}
    Cut(A_i, A_j, \cdots, A_k) = \frac{1}{2} \sum_{i=1}^{k} W(A_i, \bar{A_i})
\end{equation}
where $\bar{A_i}$ refers the supplementary set.

Minimization of Eq.\ref{equation:Loss of cut} refers the $cut$ that use \emph{Smallest} cut.
However, to get the \emph{Best} cut, it has to minimize the $RatioCut$
\begin{equation}
    \label{equation:Loss of ratio cut}
    Ratio(A_i, A_j, \cdots, A_k) = \frac{1}{2} \sum_{i=1}^{k} \frac{W(A_i, \bar{A_i})}{|A_i|}
\end{equation}
where the denominator is the sum of weights in $A_i$.

\subsection{Computation}
Direct computing $RatioCut$ is NP-hard, use clustering vectors ${h_1, h_2, \cdots, h_k}$ as alternative method
\begin{equation*}
    h_{ij} =
    \left\{
    \begin{array}{lr}
        0                      & , v_i \notin A_j \\
        \frac{1}{\sqrt{|A_j|}} & , v_i \in A_j
    \end{array}
    \right.
\end{equation*}
\end{document}