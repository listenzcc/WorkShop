\documentclass[a4paper]{article}
\usepackage{amsmath}

\begin{document}

\abstract
In a graph, known as $G(V, E)$, an important problem is how to clustering the vertex \emph{automatically}.
That means that the clustering method should be \emph{unsupervised}.
The spectral clustering method is a powerful solution.

\section {Laplacian matrix and graph}
The weight matrix is defined as
\begin{equation*}
    W = [w_{ij}]
\end{equation*}
in which, $w_{ij}=w_{ji}$ is the measurement weight of the edge, refers the distance between $V_i$ and $V_j$.

The degree matrix is diagonal matrix
\begin{equation*}
    d_{ii} = \sum_{j=1}^{N}w_{ij}
\end{equation*}
where $N$ is the number of vertex.

The subtraction is \emph{Laplacian} matrix $L$
\begin{equation}
    \label{equation: Laplacian matrix}
    L = D - W
\end{equation}

For any $N$ dimensional vector $f \in R^N$
\begin{equation}
    \label{equation: Laplacian matrix times vector}
    f^T L f = \frac{1}{2} \sum_{i, j=1}^{N} w_{ij}(f_i - f_j)^2
\end{equation}

\end{document}