\documentclass[../main.tex]{subfiles}

\begin{document}

\subsection{Gamma and Beta function}

An infinity integral is called as \emph{Gamma ($\Gamma$) function}

\begin{equation}
    \Gamma(z) = \int_{0}^{\infty} x^{z-1} e^{-x} \,\mathrm{d}x
\end{equation}

The \emph{Beta ($B$) function} is a two-factor function, derived from $\Gamma$ function

\begin{equation}
    B(\alpha, \beta) = \frac{\Gamma(\alpha) \cdot \Gamma(\beta)}{\Gamma(\alpha + \beta)}
\end{equation}

\subsection{Important equations}

\begin{proposition} \label{proposition: Gamma function propositions}
    Some very important equations.

    The value of $\Gamma(\frac{1}{2})$
    \begin{equation}
        \Gamma(\frac{1}{2}) = \sqrt {\pi}
    \end{equation}

    The recursive of $\Gamma(n)$, the general situation,
    \begin{align}
        \Gamma(1+z) & = z \Gamma(z)   \\
        \Gamma(1-z) & = -z \Gamma(-z)
    \end{align}

    The integer situation,
    \begin{equation}
        \Gamma(n) = (n-1)! \quad \forall n \in \mathcal{N}^+
    \end{equation}

    The relationship between $\Gamma$ and $e^{-x^{2}}$
    \begin{equation}
        \Gamma(z) = 2 \int_{0}^{\infty} x^{2z-1} e^{-x^{2}} \,\mathrm{d}x
    \end{equation}

    The relationship between $\Gamma$ and $B$ Function
    \begin{equation}
        B(\alpha, \beta) = \int_{0}^{1} t^{\alpha-1} (1-t)^{\beta-1} \,\mathrm{d}t
    \end{equation}

    See Lemma \ref{lemma: The relationship between Gamma and Beta} for proof.

\end{proposition}

\end{document}