\documentclass[../main.tex]{subfiles}

\begin{document}

\subsection{Determine the valid sample size}

People trend to believe the results of surveys, but are they true?
The sample size may be a key to answer the question.

% --------------------------------------------------------------------------------------
\bigbreak
\textbf{Ground Truth}

Let's start with a simple model of 'How do people support a candidate?'.
The supporting rate is a random variable that fits
\begin{equation*}
    R_{support} \sim \mathcal{N}(\mu, \sigma^2)
\end{equation*}
where $\mu$ and $\sigma^2$ are unavailable.

% --------------------------------------------------------------------------------------
\bigbreak
\textbf{Population Survey}
The supporting rate is so important that they want to know it using \emph{LARGE} surveys.
It is like asking every body.
The number of supporters follows binominal distribution
\begin{equation*}
    p(n) = (N, n) \cdot r^n \cdot (1-r)^{N-n}
\end{equation*}
where $r$ refers the ground truth of supporting rate, $N$ is the population and $n$ is the number of supporters.

Then, the sample expectation and sample variance of $n$ is direct
\begin{align}
    E(n) & = N \cdot r             \\
    D(n) & = N \cdot r \cdot (1-r)
\end{align}

Divide by $N$, we can get the \emph{unbiased estimation} \footnote{The unbiased estimation refers the sample mean of the random variable equals to the expectation.} of $r$ as $\hat{r}$ and its sample variance
\begin{align}
    \hat{r}    & = \frac{n}{N}             \\
    D(\hat{r}) & = \frac{r \cdot (1-r)}{N}
\end{align}
since $E(\hat{r})=r$.

It turns out that the variance is related to $r$ value.
There are things to remember
\begin{itemize}
    \item The variance is symmetric to $0.5$.
    \item The closer $r$ to $0.5$, the larger is the variance.
    \item The variance decreases when the sample size increases.
\end{itemize}

% --------------------------------------------------------------------------------------
\bigbreak
\textbf{Sample Survey}

However, in practice, the survey of all population is impossible.
Usually, survey in a small group (the number is $M < N$) is available.
It will turn out a similar situation
\begin{align}
    \hat{r}    & = \frac{m}{M}             \\
    D(\hat{r}) & = \frac{r \cdot (1-r)}{M}
\end{align}

\bigbreak
\textbf{Estimation}

Use above analysis, we can say that the estimation of $\mu$ value of $R_{support}$ is easy to compute.
But the real question is \emph{How we can trust the estimation?}.
Especially in the case of the survey is restricted in \emph{part} of the population.

We analysis the question on both side.

\begin{itemize}
    \item \textbf{In ground-truth end}, we have the equation that
          \begin{equation*}
              \frac{\sigma^2}{N} = \frac{s^2}{N-1}
          \end{equation*}
          where $s^2$ refers the sample variance with population of $N$.
    \item \textbf{In survey end}, we have the variance that
          \begin{align*}
              s_{N}^2 & = \frac{r(1-r)}{N} \\
              s_{M}^2 & = \frac{r(1-r)}{M}
          \end{align*}
          where $s_N^2$ and $s_M^2$ refer the variance of $N$ and $M$ population.

    \item \textbf{Jointly}, since the variance of $M$ is derived from the intrinsic variance $\sigma^2$, and population of $N$.
          It meets
          \begin{equation*}
              s_M^2 = s_N^2 + s^2
          \end{equation*}

\end{itemize}

Use the joint equation, the $M$ follows
\begin{equation*}
    M = N \cdot \frac{\frac{r(1-r)}{\sigma^2}}{(N-1)+\frac{r(1-r)}{\sigma^2}}
\end{equation*}

Under certain error edge $e$, use the equation between $e$ and $\sigma^2$
\begin{equation*}
    e^2 = z^2 \cdot \sigma^2
\end{equation*}
where $z$ is the \emph{Percentile} of normal distribution according to $e$.

Thus, we have the relationship between $e$ and $M$.
\begin{equation*}
    \hat{M} = N \cdot \frac{z^2\frac{r(1-r)}{e^2}}{(N-1)+z^2\frac{r(1-r)}{e^2}}
\end{equation*}

The solution is the minimization of \emph{Sample size} to achieve certain degree of confidence defined by $e$.

\end{document}







