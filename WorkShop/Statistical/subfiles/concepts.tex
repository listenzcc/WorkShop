\documentclass[../main.tex]{subfiles}

\begin{document}
\subsection{Random Variables}
A variable is \emph{random} means it is not fixed.
It turns out that one can obtain different values every time.
The reason behind can be systemic or arbitrary.
The aim of statistical analysis is to uncover the reason, however it usually matters little during the calculation.
But the analysis can be valid only if the \emph{random variable follows certain rules} instead of being totally unreasonable.

\subsection{Statistics and Distribution}
To understand the rules, the obtained values should be calculated carefully to formulate \emph{new meaningful values}.
The new values are called \emph{statistics}.
The statistics are asserted to be following some certain \emph{distribution}.
The distribution refers to the rule that controls the uncertainty of the random variable.
A classic distribution contains two parts:
\begin{itemize}
    \item Values $x$: The possible values of the statistic
    \item Probabilities $p(x)$: The probabilities of the values
\end{itemize}
It is also intrinsic that the sum of the probabilities should be equal to ONE, no more no less.
\begin{equation*}
    \int p(x) = 1, \forall p(x) \in (0, 1)
\end{equation*}
The function of $p(x)$ is called \emph{probability distribution function (PDF)}.

\subsection{Statistics}
There are several commonly used statistics like: \emph{expectation}, \emph{variance}, and \textit{etc}.
\begin{itemize}
    \item Expectation: The expectation value of every good obtain, expressed as the first-order zero moment
    \item Variance: The variance of the statistics, expressed as the second-order center moment
\end{itemize}
\begin{equation}
    \label{Definition: Expectation and variance definition}
    \begin{split}
        Expectation = \mathcal{E} & = \int x \cdot p(x) dx                 \\
        Variance = \mathcal{V}    & = \int (x-\mathcal{E})^2 \cdot p(x) dx
    \end{split}
\end{equation}
\begin{lemma}
    \label{Lemma: Relationship between expectation and variance}
    For simplicity, the relationship between expectation and variance can be found following
    \begin{equation*}
        \mathcal{V} = \mathcal{E}(x^2) - \mathcal{E}^2(x)
    \end{equation*}
    \begin{proof}
        Compute the square in \eqref{Definition: Expectation and variance definition},
        we have
        \begin{align*}
            \mathcal{V} & = \int (x^2 - 2 x \mathcal{E} + \mathcal{E}^2) p(x) dx \\
                        & = \mathcal{E}(x^2) - \mathcal{E}^2(x)
        \end{align*}
        where $\mathcal{E}$ refers $\mathcal{E}(x)$.
        And, the equation uses the condition that the $\mathcal{E}$ is constant in the integral.
    \end{proof}
\end{lemma}

\subsubsection{Independency of statistics}
The distribution of two random variables can be computed using \emph{joint probability} and \emph{conditional probability}.
\begin{equation*}
    p(x, y) = p(x) \cdot p(y|x) = p(y) \cdot p(x|y)
\end{equation*}

The simplest situation is the variables of $x$ and $y$ are independent with each other.
\begin{lemma}
    \label{Lemma: Confusing second-order moment of independent variables}
    If $x$ and $y$ are independent, we have $\mathcal{E}(x, y) = \mathcal{E}(x) \cdot \mathcal{E}(y)$.
    \begin{proof}
        The independency guarantees
        \begin{align*}
            p(x|y)  & = p(x)            \\
            p(y|x)  & = p(y)            \\
            p(x, y) & = p(x) \cdot p(y)
        \end{align*}
        Using the definition of expectation in \eqref{Definition: Expectation and variance definition},
        we have $\mathcal{E}(x, y) = \mathcal{E}(x) \cdot \mathcal{E}(y)$.
    \end{proof}
\end{lemma}

\subsection{Distributions}
There are several commonly used distributions like: \emph{Normal distribution}, \emph{Binomial distribution}, \emph{Chi-squared distribution}, \emph{Student's t-distribution} and \textit{etc}.

\bigbreak
The PDF of Normal distribution is
\begin{equation}
    \label{Definition: PDF of Normal distribution}
    p(x)=
    \frac{1}{\sqrt{2\pi}\sigma} \exp{\frac{(x-\mu)^2}{2\sigma^2}},
    x \in (-\infty, \infty)
\end{equation}
where $\mathcal{E}=\mu$ and $\mathcal{V}=\sigma^2$.
The normal distribution is so important that we express it as $p(x) \sim \mathcal{N}(\mu, \sigma^2)$.

\bigbreak
The PDF of Binomial distribution is
\begin{equation}
    \label{Definition: PDF of Binomial distribution}
    p_N(n)=
    (N, n) \cdot r^n \cdot (1-r)^{N-n},
    n \in [0, N]
\end{equation}
where $\mathcal{E}=N \cdot r$ and $\mathcal{V}=N \cdot r \cdot (1-r)$.

\bigbreak
The PDF of Chi-squared distribution is
\begin{equation}
    \label{Definition: PDF of Chi-squared distribution}
    p_r(x)=
    \frac{x^{r/2-1} e^{-x/2}}{\Gamma(r/2) 2^{r/2}},
    x \in (0, \infty)
\end{equation}
where $\mathcal{E}=r$ and $\mathcal{V}=2r$.

\subsection{Parameter Estimation}
One goal of statistical analysis is to \emph{determine the parameters of the distribution}.
There are several methods of the estimation:
\begin{itemize}
    \item MLE: Maximum Likelihood Estimation
    \item MAP: Maximum A posteriori Probability estimate
\end{itemize}

\end{document}
