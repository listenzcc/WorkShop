\documentclass[../concepts.tex]{subfiles}

\begin{document}
\subsubsection{Explain of Lemma\ref{Lemma: Relationship between expectation and variance}}
\textbf{Relationship between expectation and variance}
% ---------------------------------------
%    Original contents
\begin{quote}
    For simplicity, the relationship between expectation and variance can be found as following
    \begin{equation*}
        \mathcal{V} = \mathcal{E}(X^2) - \mathcal{E}^2(X)
    \end{equation*}
\end{quote}
% ---------------------------------------

% ---------------------------------------
%    Your word goes here

For simplicity, the relationship between expectation and variance can be found as following
\begin{equation*}
    \mathcal{V} = \mathcal{E}(X^2) - \mathcal{E}^2(X)
\end{equation*}


Prove that, the relationship between expectation and variance can be found as following
\begin{equation*}
    \mathcal{V} = \mathcal{E}(X^2) - \mathcal{E}^2(X)
\end{equation*}

\begin{proof}
    Compute the square in \eqref{Definition: Expectation and variance definition},
    we have
    \begin{align*}
        \mathcal{V} & = \int (x^2 - 2 x \mathcal{E} + \mathcal{E}^2) p(x) dx \\
                    & = \mathcal{E}(X^2) - \mathcal{E}^2(X)
    \end{align*}
    where $\mathcal{E}$ refers $\mathcal{E}(X)$.
    And, the equation uses the condition that the $\mathcal{E}$ is constant in the integral.
\end{proof}
% ---------------------------------------
\end{document}
